\documentclass[a4paper,11pt]{article}
\usepackage[spanish]{babel}
\usepackage[utf8]{inputenc}

% Configuración páginas
\usepackage{vmargin}							% Márgenes

\usepackage{sectsty}							% Fuente de los títulos
\allsectionsfont{\normalfont \Large \scshape}

\usepackage{graphicx}							% Imágenes
\graphicspath{{images/}}

\usepackage{listliketab}						% Tratar indentacion distas como tablas
\usepackage{tabularx}
\usepackage{multirow}							% Ocupar multiples filas en tablas

\usepackage{mathtools}							% Matematicas
\newcommand\numberthis{							% numeración en align*
	\addtocounter{equation}{1}\tag{\theequation}
}

% Configuración del título
\newcommand{\horrule}[1]{\rule{\linewidth}{#1}} 	% Horizontal rule

\title{
	\vspace{-25pt}
	\normalfont \Large \textsc{
		Modelos de Investigación Operativa,
        Ingeniería Informática\\
        Universidad de Valladolid
	}\\[10pt]
	\horrule{1pt}\\[10pt]
	\huge \textbf{
		Práctica 7
	}\\
	\horrule{1pt}
}
\author{
	\normalfont \Large Daniel González Alonso
}
\date{
	\normalfont \large \today
}

%%%%%%%%%%%%%%%%%%%%%%%%%%%%%%%%%%%%%%%%%%%%%%%%%%
\begin{document}
\maketitle

%%%% RESUMEN %%%%
\begin{abstract}
	En este documento se describen los problemas y los resultados obtenidos de la práctica 7 del tema 3 de la asignatura Modelos de Investigación Operativa de Ingeniería Informática, Universidad de Valladolid.
\end{abstract}

%%%% DESARROLLO %%%%
\section{Introducción}
%%%%%%%%%%%%%%%%%%%%%%%%%%%%%%%%%%%%%%%%%%%%%%%%%%%%%%
Esta práctica trata de problemas de cubrimiento máximo. El modelo empleado para resolver estos problemas es el siguiente:

\begin{align*}\numberthis
   	&\textrm{Maximizar }	&& \sum_{i=1}^{m}{h_i \cdot z_i}		&&				\\
   	&\textrm{Sujeto a }		&& \sum_{j \in N_{i}}{x_j} \geq z_i 	&& i=1,\ldots,m \\
    &						&& \sum_{j=1}^{n}{x_j} \leq P			&&				\\
	&						&& x_{j} \in \{0,1\}					&& j=1,\ldots,n \\
	&						&& z_{i} \in \{0,1\}					&& i=1,\ldots,m \\
\end{align*}

Donde ${x_j}$ representa si se abre una instalación en ${j}$ (1) o no (0), ${z_i}$ nos dice si la demanda de ${i}$ queda cubierta (1) o no (0), ${h_i}$ es la demanda de ${i}$ y ${P}$ es el número de instalaciones que se van a abrir. El objetivo del modelo como se puede observar es maximizar la demanda que queda cubierta para un cierto número de instalaciones.

\newpage
\section{Desarrollo}

Para esta práctica se nos pide resolver el problema de cubrimiento máximo para los problemas \texttt{data/aint1.dat} y \texttt{data/aint5.dat} mediante el método exacto, el método greedy y el método greedy aleatorizado (${K=5}$ y ${N=100}$) para valores de ${p}$ entre 1 y 6 y una distancia de cubrimiento ${dc=200}$ (equivalente a 20 Km). Al final se nos pide que para cada método construyamos las tablas o gráficas con la demanda cubierta y el porcentaje de demanda cubierta para cada valor de ${p}$.\\

Esta práctica se encuentra resuelta mediante \textit{Xpress Mossel} en los archivos \texttt{max\_covering \_aint1.mos} y \texttt{max\_covering\_aint5.mos} (el nombre indica el fichero de datos empleado). En cada uno de los ficheros se resuelve el problema con los métodos pedidos.\\

Para resolver esta práctica en cada uno de los ficheros lo primero que se hizo fue cargar los datos. El formato de los datos \textit{aint} es el siguiente:

\storestyleof{itemize}
\begin{listliketab}
    \begin{tabular}{Lll}
\textbullet	& ${m}$							& Número de puntos de demanda	\\
\textbullet	& ${n}$ 						& Número de puntos de servicio	\\
\textbullet	& ${h_i		\ \ i=1,\ldots,m}$	& Demandas, una por línea		\\
\textbullet	& ${d_{i,j}	\ \ i=1,\ldots,m \ \ j=1,\ldots,n}$	& Matriz ${m \times n}$ de distancias (Hectómetros)	\\
    \end{tabular}
\end{listliketab}

Una vez cargados los datos, la siguiente parte consiste en obtener la resolución exacta. La solución exacta obtenida con Xpress no tiene mayor dificultad que introducir el modelo anterior, lo único destacable de esta parte fue el tiempo. La solución de Xpress nos dio 0.516 segundos para los datos \texttt{aint1} y 0.838 segundos para \texttt{aint5} después de ejecutarlo con todos los valores de P.\\

En la siguiente parte se hizo la solución greedy. El algoritmo se describe a continuación:

\begin{enumerate}
\item Buscamos el indice del nodo que cubre la mayor cantidad de demanda y que no esté fijado todavía. Como detalles de implementación cabe destacar que en mi caso, para este paso simplemente calculaba la suma de toda la demanda que cubre un punto (los puntos que cubre cada nodo de servicio están precalculados en una matriz llamada \texttt{cubrimientos}), y con ello buscaba el máximo, cuyo indice es llamado \texttt{j\_max}. Además en caso de que en un nodo de servicio la suma de demanda que puede cubrir diese 0, lo fijo y añado a la solución con un 0.
\item Fijamos el nodo anterior y los añadimos a la solución.
\item Eliminamos la demanda ya cubierta. En mi caso este paso lo hacía simplemente poniendo un 0 en el vector de demandas (llamado ${demandas}$) en todos aquellos nodos de de demanda que se encuentren cubiertos por \texttt{j\_max}.
\item Si todavía no hemos cubierto toda la demanda y el número de instalaciones es inferior a ${p}$ volvemos al paso 1.
\end{enumerate}

La solución greedy obtenida es en este caso la más rápida, con 0.031 segundos para los datos \texttt{aint1} y también de 0.031 segundos para \texttt{aint5} después de ejecutarlo con todos los valores de P.\\

Por último se hizo la versión greedy aleatorizada, la cual se diferencia del algoritmo greedy ``normal'' en que para el primer paso se dispone de una lista \textit{RCL} con los ${K}$ indices (5 en nuestro caso) de los mejores valores en la iteración actual, y de entre ellos se selecciona uno aleatoriamente para así obtener ${j\_max}$. En mi caso para hacer esta parte cree un vector para la lista RCL de tamaño ${K}$ y otro con con banderas para indicar si un elemento ya se encontraba en la lista RCL o no el cual reiniciaba en cada iteración. El calculo de cada elemento de la lista RCL se hace de forma idéntica al algoritmo greedy sin \textit{RCL}, solo que hay que tener en cuenta si un elemento ya está marcado (en la RCL) o no. Además de esta lista \textit{RCL} hay que destacar en el algoritmo greedy aleatorizado el bucle anterior ejecutarse ${N}$ veces (en nuestro caso 100), y en cada una de las iteraciones se comprueba si la solución es obtenida es mejor que la de las iteraciones anteriores.\\

Sobre el tiempo de ejecución, el algoritmo greedy aleatorizado es sin duda alguna el más lento, tardando 14.529 segundos para \texttt{aint1} y 12.519 segundos para \texttt{aint5}.

\section{Resultados}

\begin{itemize}
\item Resultados obtenidos para \texttt{aint1}:

\begin{table}[!htbp]
\label{results_aint1}
\centering
\begin{tabularx}{\textwidth}{|p{1.5cm}|p{1.5cm}|p{2cm}|p{1.5cm}|p{2cm}|p{1.5cm}|p{2cm}|}
\hline
\multirow{2}{\hsize}{Valor de ${P}$}	& \multicolumn{2}{X|}{Método Exacto}	& \multicolumn{2}{X|}{Algoritmo Greedy}	& \multicolumn{2}{X|}{Algoritmo Greedy Aleatorizado} \\ \cline{2-7}
& Demanda Cubierta  & Porcentaje de la demanda Cubierta	& Demanda Cubierta  & Porcentaje de la demanda Cubierta	& Demanda Cubierta  & Porcentaje de la demanda Cubierta	\\ \hline
1	& 15397	& 83.1506\%	& 15397	& 83.1506\%	& 15397	& 83.1506\%	\\ \hline
2	& 17055	& 92.1046\%	& 16746	& 90.4358\%	& 16898	& 91.2567\%	\\ \hline
3	& 17912	& 96.7327\%	& 17802	& 96.1387\%	& 17912	& 96.7327\%	\\ \hline
4	& 18411	& 99.4276\%	& 18230	& 98.4501\%	& 18320	& 98.9361\%	\\ \hline
5	& 18517	& 100\%		& 18427	& 99.514\%	& 18517	& 100\%		\\ \hline
6	& 18517	& 100\%		& 18517	& 100\%		& 18517	& 100\%		\\ \hline
\end{tabularx}
\end{table}

\item Resultados obtenidos para \texttt{aint5}:

\begin{table}[!htbp]
\label{results_aint5}
\centering
\begin{tabularx}{\textwidth}{|p{1.5cm}|p{1.5cm}|p{2cm}|p{1.5cm}|p{2cm}|p{1.5cm}|p{2cm}|}
\hline
\multirow{2}{\hsize}{Valor de ${P}$}	& \multicolumn{2}{X|}{Método Exacto}	& \multicolumn{2}{X|}{Algoritmo Greedy}	& \multicolumn{2}{X|}{Algoritmo Greedy Aleatorizado} \\ \cline{2-7}
& Demanda Cubierta  & Porcentaje de la demanda Cubierta	& Demanda Cubierta  & Porcentaje de la demanda Cubierta	& Demanda Cubierta  & Porcentaje de la demanda Cubierta	\\ \hline
1	& 8405	& 88.8854\%	& 8405	& 88.8854\%	& 8405	& 88.8854\%	\\ \hline
2	& 9134	& 96.5948\%	& 9097	& 96.2035\%	& 9117	& 96.415\%	\\ \hline
3	& 9365	& 99.0376\%	& 9312	& 98.4772\%	& 9337	& 98.7415\%	\\ \hline
4	& 9456	& 100\%		& 9404	& 99.4501\%	& 9429	& 99.7145\%	\\ \hline
5	& 9456	& 100\%		& 9456	& 100\%		& 9456	& 100\%		\\ \hline
6	& 9456	& 100\%		& 9456	& 100\%		& 9456	& 100\%		\\ \hline
\end{tabularx}
\end{table}

\end{itemize}

\end{document}
