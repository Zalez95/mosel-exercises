\documentclass[a4paper,11pt]{article}
\usepackage[spanish]{babel}
\usepackage[utf8]{inputenc}

% Configuración páginas
\usepackage{vmargin}				% Márgenes

\usepackage{sectsty}				% Fuente de los títulos
\allsectionsfont{\normalfont \Large \scshape}

\usepackage{graphicx}				% Imágenes
\graphicspath{{images/}}

\usepackage{mathtools}				% Matematicas
\newcommand\numberthis{				% numeración en align*
	\addtocounter{equation}{1}\tag{\theequation}
}

% Configuración del título
\newcommand{\horrule}[1]{\rule{\linewidth}{#1}} 	% Horizontal rule

\title{
	\vspace{-25pt}
	\normalfont \Large \textsc{
		Modelos de Investigación Operativa,
        Ingeniería Informática\\
        Universidad de Valladolid
	}\\[10pt]
	\horrule{1pt}\\[10pt]
	\huge \textbf{
		Práctica 3
	}\\
	\horrule{1pt}
}
\author{
	\normalfont \Large Daniel González Alonso
}
\date{
	\normalfont \large \today
}

%%%%%%%%%%%%%%%%%%%%%%%%%%%%%%%%%%%%%%%%%%%%%%%%%%
\begin{document}
\maketitle

%%%% RESUMEN %%%%
\begin{abstract}
	En este documento se describen los problemas y los resultados obtenidos de la práctica 3 del tema 2 de la asignatura Modelos de Investigación Operativa de Ingeniería Informática, Universidad de Valladolid.
\end{abstract}

%%%% DESARROLLO %%%%
\section{Introducción}
Esta práctica trata del calculo del problema de Set Covering con datos dados en un array disperso. El modelo empleado para este problema es el mismo que el usado ya en las anteriores prácticas 1 y 2.

\section{Ejemplo de Sayre-Priors}

El problema consiste en determinar de los ${n=37}$ posibles planes de vuelo, cuales escoger si solo disponemos de ${m=10}$ tripulaciones. Para ello se nos da de datos ${a_{i,j}}$, donde cada casilla puede valer 1 si ${i}$ puede ser a empleado en el plan de vuelo ${j}$ o 0 en caso contrario. También se nos da los valores de los costes de cada plan de vuelo ${f_j}$.\\

Para resolver este problema primero se tuvo que almacenar estos datos en el fichero \texttt{3\_crew\_shed2.dat}, cabe destacar que la matriz ${a_{i,j}}$ al ser muy grande y dispersa, solo se han almacenado aquellos valores que valgan 1. El problema fue resuelto en \textit{Xpress Mosel} en el fichero \texttt{3\_crew\_shed2.mos}. Los resultados obtenidos fueron que el mínimo coste es de 9.000\$ si se empleaban los planes de vuelo 12, 15, 25 y 32.

\end{document}
